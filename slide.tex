% Created 2023-06-20 Tue 17:50
% Intended LaTeX compiler: pdflatex
\documentclass[fleqn,aspectratio=1610]{beamer}
\author{Noboru Murata}
\date{\today}
\title{Time-Varying Transition Probability Matrix}
\subtitle{application to brand share analysis}
\usepackage[toc=none]{mytalk}
\addbibresource{papers.bib}
\graphicspath{{figs/}{refs/}}
\DeclareGraphicsExtensions{.pdf,.png,.eps,.jpg}
\institute{\url{https://noboru-murata.github.io/}}
\hypersetup{
 pdfauthor={Noboru Murata},
 pdftitle={Time-Varying Transition Probability Matrix},
 pdfkeywords={transition probability matrix, brand share},
 pdfsubject={based on Chiba et al (2017), https://doi.org/10.1371/journal.pone.0169981},
 pdfcreator={Emacs 28.2 (Org mode 9.6.4)}, 
 pdflang={English}}
\begin{document}

\maketitle
\begin{frame}{Outline}
\tableofcontents
\end{frame}


\section{Introduction}
\label{sec:org901277c}
\subsection{motivated examples}
\label{sec:orgba54c54}
\begin{frame}[label={sec:orga0eb2ac}]{Graph Structure Analysis}
two problems in graph inference:
\begin{itemize}
\item \alert{direct problem (deductive)}: \\[0pt]
induce properties from known graph structure
\begin{itemize}
\item shortest path, traveling salesman, graph coloring \%problem
\item maximum flow, minimum cut \%problem
\item Google search (stationary distribution of Google matrix)
\end{itemize}
\item \alert{inverse problem (inductive)}: \\[0pt]
estimate graph structure from partially observed properties
\begin{itemize}
\item path analysis, graphical model estimation
\item sparse estimation of accuracy matrix
\item structure estimation via graph Laplacian
\end{itemize}
\end{itemize}
\end{frame}

\begin{frame}[label={sec:org8d344e3}]{Motivated Problem}
\begin{columns}
\begin{column}{0.6\columnwidth}
\begin{center}
{\small automobile sales data of manufacturers}\\
\includegraphics[page=4,
trim=40 285 30 80, clip,
width=0.95\linewidth]
{Chiba_etal2017}
\end{center}
\end{column}
\begin{column}{0.4\columnwidth}
\begin{alertblock}{Questions}
\begin{itemize}
\item why sales shares vary?
\item what happens in customer preferences?
\end{itemize}
\end{alertblock}
\end{column}
\end{columns}
\end{frame}

\subsection{stationary distribution of Google matrix}
\label{sec:orge03db20}
\begin{frame}[label={sec:org8ab313f}]{Running Example}
\begin{columns}
\begin{column}{0.65\columnwidth}
\begin{center}
\includegraphics[page=1,width=.9\linewidth]{weightedgraph}
\end{center}
\end{column}
\begin{column}{0.35\columnwidth}
directed graph
\begin{itemize}
\item number of node: 26
\item edge exist.: 0.1
\item weight: \\[0pt]
uniform on \([0,1]\)
\end{itemize}
\end{column}
\end{columns}
\end{frame}

\begin{frame}[label={sec:orgb85c226}]{Adjacency Matrix}
\begin{itemize}
\item adjacency matrix \(W\)
\begin{equation}
  (W)_{ij} = \text{strength of connection from \(i\) to \(j\)}
\end{equation}
\item indicate vector of sink node \(\boldsymbol{a}\)
\begin{equation}
  (\boldsymbol{a})_i =
  \begin{cases}
    1, & \text{if}\;(W\boldsymbol{e})_i = 0\\
    0, & \text{otherwise}
  \end{cases}
\end{equation}
where \(\boldsymbol{e}\) is a vector of all \(1\)
\item normalized adjacency matrix (transition matrix) \(H\) 
\begin{equation}
  H = \mathrm{diag}(W\boldsymbol{e}+\boldsymbol{a})^{-1} W
\end{equation}
\((H\boldsymbol{e})_i=1\) holds except for sink nodes
\end{itemize}
\end{frame}

\begin{frame}[label={sec:org0ff587a}]{Google Matrix}
\begin{itemize}
\item transition matrix with sink node adjustment \(S\)
\begin{align}
  S =& H + \boldsymbol{a}\boldsymbol{e}^T/n\\
  =&\text{(probabilistic transition)}\\
     &+\text{(escape from sink nodes)}
\end{align}
probability matrix: \(S\boldsymbol{e}=\boldsymbol{e}\)
\item Google matrix \(G\)
\begin{align}
  G
  =& \alpha S + (1-\alpha) \boldsymbol{e}\boldsymbol{e}^T/n\\
  =&\text{(transition along edges)}
     +\text{(random transition)}
     % &=\underbrace{\text{(第1項)}}_{\text{隣接行列による確率遷移}} 
     % +\underbrace{\text{(第2項)}}_{\text{ランダムサーフによる確率遷移}}\\
\end{align}
probability matrix: \(G\boldsymbol{e}=\boldsymbol{e}\)
\end{itemize}
\end{frame}

\begin{frame}[label={sec:orga84ff77}]{}
\begin{columns}
\begin{column}{0.6\columnwidth}
\includegraphics<+>[page=1,width=1.0\linewidth]{statdist}%
\includegraphics<+>[page=2,width=1.0\linewidth]{statdist}%
\includegraphics<+>[page=3,width=1.0\linewidth]{statdist}%
\includegraphics<+>[page=4,width=1.0\linewidth]{statdist}%
\includegraphics<+>[page=5,width=1.0\linewidth]{statdist}%
\includegraphics<+>[page=6,width=1.0\linewidth]{statdist}%
\includegraphics<+>[page=7,width=1.0\linewidth]{statdist}%
\includegraphics<+>[page=8,width=1.0\linewidth]{statdist}%
\includegraphics<+>[page=9,width=1.0\linewidth]{statdist}%
\includegraphics<+>[page=10,width=1.0\linewidth]{statdist}%
\includegraphics<+>[page=11,width=1.0\linewidth]{statdist}%
\includegraphics<+>[page=12,width=1.0\linewidth]{statdist}%
\includegraphics<+>[page=13,width=1.0\linewidth]{statdist}%
\end{column}
\begin{column}{0.4\columnwidth}
\begin{itemize}
\item \(\alpha=0.85\)
\item initial: $\backslash$\ uniform dist.
\end{itemize}
\end{column}
\end{columns}
\end{frame}

\begin{frame}[label={sec:org1ed8c43}]{}
\begin{columns}
\begin{column}{0.6\columnwidth}
\includegraphics<+>[page=14,width=1.0\linewidth]{statdist}%
\includegraphics<+>[page=15,width=1.0\linewidth]{statdist}%
\includegraphics<+>[page=16,width=1.0\linewidth]{statdist}%
\includegraphics<+>[page=17,width=1.0\linewidth]{statdist}%
\includegraphics<+>[page=18,width=1.0\linewidth]{statdist}%
\includegraphics<+>[page=19,width=1.0\linewidth]{statdist}%
\includegraphics<+>[page=20,width=1.0\linewidth]{statdist}%
\includegraphics<+>[page=21,width=1.0\linewidth]{statdist}%
\includegraphics<+>[page=22,width=1.0\linewidth]{statdist}%
\includegraphics<+>[page=23,width=1.0\linewidth]{statdist}%
\includegraphics<+>[page=24,width=1.0\linewidth]{statdist}%
\includegraphics<+>[page=25,width=1.0\linewidth]{statdist}%
\includegraphics<+>[page=26,width=1.0\linewidth]{statdist}%
\end{column}
\begin{column}{0.4\columnwidth}
\begin{itemize}
\item \(\alpha=0.85\)
\item initial: \\[0pt]

12 nodes
\end{itemize}
\end{column}
\end{columns}
\end{frame}

\begin{frame}[label={sec:org1b76400}]{Property of Stationary Distribution}
\begin{columns}
\begin{column}{0.5\columnwidth}
\begin{center}
initial condition 1\\
\includegraphics<1>[page=12,width=.95\linewidth]{statdist}%
\includegraphics<2>[page=13,width=.95\linewidth]{statdist}%
\end{center}
\end{column}
\begin{column}{0.5\columnwidth}
\begin{center}
initial condition 2\\
\includegraphics<1>[page=25,width=.95\linewidth]{statdist}%
\includegraphics<2>[page=26,width=.95\linewidth]{statdist}%
\end{center}
\end{column}
\end{columns}
\begin{center}
fast convergence to the same stationary distribution
\end{center}
\end{frame}

\begin{frame}[label={sec:orgae0c58d}]{}
\begin{columns}
\begin{column}{0.6\columnwidth}
\includegraphics<+>[page=27,width=1.0\linewidth]{statdist}%
\includegraphics<+>[page=28,width=1.0\linewidth]{statdist}%
\includegraphics<+>[page=29,width=1.0\linewidth]{statdist}%
\includegraphics<+>[page=30,width=1.0\linewidth]{statdist}%
\includegraphics<+>[page=31,width=1.0\linewidth]{statdist}%
\includegraphics<+>[page=32,width=1.0\linewidth]{statdist}%
\includegraphics<+>[page=33,width=1.0\linewidth]{statdist}%
\includegraphics<+>[page=34,width=1.0\linewidth]{statdist}%
\includegraphics<+>[page=35,width=1.0\linewidth]{statdist}%
\includegraphics<+>[page=36,width=1.0\linewidth]{statdist}%
\includegraphics<+>[page=37,width=1.0\linewidth]{statdist}%
\includegraphics<+>[page=38,width=1.0\linewidth]{statdist}%
\includegraphics<+>[page=39,width=1.0\linewidth]{statdist}%
\end{column}
\begin{column}{0.4\columnwidth}
\begin{itemize}
\item \(\alpha=0.5\)
\item initial: \\[0pt]
uniform dist.
\end{itemize}
\end{column}
\end{columns}
\end{frame}

\begin{frame}[label={sec:org506a1a3}]{Property of Scaling (\(\alpha\))}
\begin{columns}
\begin{column}{0.5\columnwidth}
\begin{center}
\(\alpha=0.85\)\\
\includegraphics<1>[page=12,width=.95\linewidth]{statdist}%
\includegraphics<2>[page=13,width=.95\linewidth]{statdist}%
\end{center}
\end{column}
\begin{column}{0.5\columnwidth}
\begin{center}
\(\alpha=0.5\)\\
\includegraphics<1>[page=38,width=.95\linewidth]{statdist}%
\includegraphics<2>[page=39,width=.95\linewidth]{statdist}%
\end{center}
\end{column}
\end{columns}
\begin{center}
random diffusion with smaller scaling
\end{center}
\end{frame}

\begin{frame}[label={sec:org30da231}]{}
\begin{columns}
\begin{column}{0.6\columnwidth}
\includegraphics<+>[page=40,width=1.0\linewidth]{statdist}%
\includegraphics<+>[page=41,width=1.0\linewidth]{statdist}%
\includegraphics<+>[page=42,width=1.0\linewidth]{statdist}%
\includegraphics<+>[page=43,width=1.0\linewidth]{statdist}%
\includegraphics<+>[page=44,width=1.0\linewidth]{statdist}%
\includegraphics<+>[page=45,width=1.0\linewidth]{statdist}%
\includegraphics<+>[page=46,width=1.0\linewidth]{statdist}%
\includegraphics<+>[page=47,width=1.0\linewidth]{statdist}%
\includegraphics<+>[page=48,width=1.0\linewidth]{statdist}%
\includegraphics<+>[page=49,width=1.0\linewidth]{statdist}%
\includegraphics<+>[page=50,width=1.0\linewidth]{statdist}%
\includegraphics<+>[page=51,width=1.0\linewidth]{statdist}%
\includegraphics<+>[page=52,width=1.0\linewidth]{statdist}%
\end{column}
\begin{column}{0.4\columnwidth}
\begin{itemize}
\item without sink node escape
\item initial: \\[0pt]
uniform dist.
\end{itemize}
\end{column}
\end{columns}
\end{frame}

\begin{frame}[label={sec:org4e8db1d}]{Property of Sink Node}
\begin{columns}
\begin{column}{0.5\columnwidth}
\begin{center}
with sink node adjustment\\
\includegraphics<1>[page=12,width=.95\linewidth]{statdist}%
\includegraphics<2>[page=13,width=.95\linewidth]{statdist}%
\end{center}
\end{column}
\begin{column}{0.5\columnwidth}
\begin{center}
without sink node adjustment\\
\includegraphics<1>[page=51,width=.95\linewidth]{statdist}%
\includegraphics<2>[page=52,width=.95\linewidth]{statdist}%
\end{center}
\begin{center}
concentration on a specific node (``sink'')
\end{center}
\end{column}
\end{columns}
\end{frame}

\begin{frame}[label={sec:orga4f7f4a}]{Applications}
a simple and strong model of movements on directed graph:
\begin{itemize}
\item behavior model of selection from finite options
\begin{itemize}
\item web surf model
\item purchase model of certain genres
\item transition model of audience ratings
\item customer share model of restaurants/coffee shops
\end{itemize}
\item adjustment of transition matrix (sink node/alpha)
\begin{itemize}
\item out of stock or service
\item capricious or adventurous attempts
\item introduction from others
\end{itemize}
\end{itemize}
\end{frame}

\section{Problem Formulation}
\label{sec:orgd3db073}
\subsection{time-varying graph and transition matrix}
\label{sec:org49f1525}
\begin{frame}[label={sec:orgaf7275e}]{Time-Varying Graph}
non-stationary data on directed graphs:
\begin{itemize}
\item strength of edges slowly change in time
\begin{itemize}
\item change of structure
\item change of stationary distribution
\end{itemize}
\item model assumption:
\begin{itemize}
\item frequent update (fast time scale; \(t\))\\[0pt]
e.g.: purchase every day
\item sparse observation
(slow time scale; \(T\))\\[0pt]
e.g.: aggregate every week
\end{itemize}
observations are supposed to be
on stationary distribution at current point
\end{itemize}
\end{frame}

\begin{frame}[label={sec:org585da73}]{}
\begin{columns}
\begin{column}{0.6\columnwidth}
\includegraphics<+>[page=1,width=1.0\linewidth]{graphdrift}%
\includegraphics<+>[page=2,width=1.0\linewidth]{graphdrift}%
\includegraphics<+>[page=3,width=1.0\linewidth]{graphdrift}%
\includegraphics<+>[page=4,width=1.0\linewidth]{graphdrift}%
\includegraphics<+>[page=5,width=1.0\linewidth]{graphdrift}%
\includegraphics<+>[page=6,width=1.0\linewidth]{graphdrift}%
\includegraphics<+>[page=7,width=1.0\linewidth]{graphdrift}%
\includegraphics<+>[page=8,width=1.0\linewidth]{graphdrift}%
\includegraphics<+>[page=9,width=1.0\linewidth]{graphdrift}%
\includegraphics<+>[page=10,width=1.0\linewidth]{graphdrift}%
\includegraphics<+>[page=11,width=1.0\linewidth]{graphdrift}%
\includegraphics<+>[page=12,width=1.0\linewidth]{graphdrift}%
\end{column}
\begin{column}{0.4\columnwidth}
change of structure
\begin{itemize}
\item 10\% edges at random
\item relatively small
\end{itemize}
\end{column}
\end{columns}
\end{frame}

\begin{frame}[label={sec:org3bed07f}]{}
\begin{center}
\onslide*<1>{
  \includegraphics[page=1,width=.32\textwidth]{graphdrift} 
  \includegraphics[page=2,width=.32\textwidth]{graphdrift}
  \includegraphics[page=3,width=.32\textwidth]{graphdrift}\\
  \includegraphics[page=4,width=.32\textwidth]{graphdrift}
  \includegraphics[page=5,width=.32\textwidth]{graphdrift}
  \includegraphics[page=6,width=.32\textwidth]{graphdrift}
}
\onslide*<2>{
  \includegraphics[page=7,width=.32\textwidth]{graphdrift}
  \includegraphics[page=8,width=.32\textwidth]{graphdrift}
  \includegraphics[page=9,width=.32\textwidth]{graphdrift}\\
  \includegraphics[page=10,width=.32\textwidth]{graphdrift}
  \includegraphics[page=11,width=.32\textwidth]{graphdrift}
  \includegraphics[page=12,width=.32\textwidth]{graphdrift}
}
\end{center}
\end{frame}

\begin{frame}[label={sec:orgb5f0ac4}]{}
\begin{columns}
\begin{column}{0.6\columnwidth}
\includegraphics<+>[page=1,width=1.0\linewidth]{driftstat}%
\includegraphics<+>[page=2,width=1.0\linewidth]{driftstat}%
\includegraphics<+>[page=3,width=1.0\linewidth]{driftstat}%
\includegraphics<+>[page=4,width=1.0\linewidth]{driftstat}%
\includegraphics<+>[page=5,width=1.0\linewidth]{driftstat}%
\includegraphics<+>[page=6,width=1.0\linewidth]{driftstat}%
\includegraphics<+>[page=7,width=1.0\linewidth]{driftstat}%
\includegraphics<+>[page=8,width=1.0\linewidth]{driftstat}%
\includegraphics<+>[page=9,width=1.0\linewidth]{driftstat}%
\includegraphics<+>[page=10,width=1.0\linewidth]{driftstat}%
\includegraphics<+>[page=11,width=1.0\linewidth]{driftstat}%
\includegraphics<+>[page=12,width=1.0\linewidth]{driftstat}%
\end{column}
\begin{column}{0.4\columnwidth}
change of stationary dist.\%ribution
\begin{itemize}
\item 10\% edges at random
\item large effect
\end{itemize}
\end{column}
\end{columns}
\end{frame}

\begin{frame}[label={sec:org5fc9fd2}]{}
\begin{center}
\onslide*<1>{
  \includegraphics[page=1,width=.32\textwidth]{driftstat} 
  \includegraphics[page=2,width=.32\textwidth]{driftstat}
  \includegraphics[page=3,width=.32\textwidth]{driftstat}\\
  \includegraphics[page=4,width=.32\textwidth]{driftstat}
  \includegraphics[page=5,width=.32\textwidth]{driftstat}
  \includegraphics[page=6,width=.32\textwidth]{driftstat}
}
\onslide*<2>{
  \includegraphics[page=7,width=.32\textwidth]{driftstat}
  \includegraphics[page=8,width=.32\textwidth]{driftstat}
  \includegraphics[page=9,width=.32\textwidth]{driftstat}\\
  \includegraphics[page=10,width=.32\textwidth]{driftstat}
  \includegraphics[page=11,width=.32\textwidth]{driftstat}
  \includegraphics[page=12,width=.32\textwidth]{driftstat}
}
\end{center}
\end{frame}

\begin{frame}[label={sec:org27fa47e}]{}
\begin{columns}
\begin{column}{0.6\columnwidth}
\begin{center}
\includegraphics[page=13,width=1.0\linewidth]{driftstat}
\end{center}
\end{column}
\begin{column}{0.4\columnwidth}
change of statinary dist.\%ribution
\begin{itemize}
\item 10\% edges at random
\item large effect
\end{itemize}
\end{column}
\end{columns}
\end{frame}
\subsection{graph estimation problem}
\label{sec:org6eb434b}
\begin{frame}[label={sec:orgad24a96}]{Our Problem}
\begin{alertblock}{Problem}
for given series of stationary distributions \(\{\pi_t\}\),
estimate series of graph structures \(\{G_t\}\).
\begin{equation}
  \text{minimize } L(\{G_t\}) \quad
  \text{subject to}\;\forall t,\;\pi_t^T G_t = \pi_t^T
\end{equation}
\end{alertblock}
difficulties:
\begin{itemize}
\item infinitely many matrices have
the same eigenvector
\item the followings are needed:
\begin{itemize}
\item assumptions on graph structures
\item assumptions of graph changes
\end{itemize}
\end{itemize}
e.g. (fused lasso):
for a certain sparse norm of matrix, \(\|\cdot\|_s\),
\begin{equation}
  L(\{G_t\}) = \sum_t \|G_t\|_s + \sum_t \|G_{t+1}-G_t\|_s 
\end{equation}
\end{frame}

\begin{frame}[label={sec:orgb66de09}]{}
\begin{center}
\includegraphics<+>[page=1,width=.9\linewidth]{schematic}%
\includegraphics<+>[page=2,width=.9\linewidth]{schematic}%  
\includegraphics<+>[page=3,width=.9\linewidth]{schematic}% 
\includegraphics<+>[page=4,width=.9\linewidth]{schematic}%  
\includegraphics<+>[page=5,width=.9\linewidth]{schematic}% 
\includegraphics<+>[page=6,width=.9\linewidth]{schematic}%  
\end{center}
\end{frame}

\begin{frame}[label={sec:org2abd949}]{Restriction of Eigenvector}
make matrices keep in a restricted subspace:
\begin{itemize}
\item eigenvector of matrix \(A\) to be
\(\boldsymbol{a}\) (unit vector):
\begin{equation}
  A\boldsymbol{a}=\lambda \boldsymbol{a}+\boldsymbol{b}
  \Rightarrow A-\boldsymbol{b}\boldsymbol{a}^T \rightarrow A
  % \Rightarrow A\leftarrow A-\boldsymbol{b}\boldsymbol{a}^T
\end{equation}
\item eigenvalue of eigenvector \(\boldsymbol{a}\)
to be \(\mu\):
\begin{equation}
  A\boldsymbol{a}=\lambda \boldsymbol{a}
  \Rightarrow A+(\mu-\lambda)\boldsymbol{a}\boldsymbol{a}^T \rightarrow A
  % \Rightarrow A\leftarrow A+(\mu-\lambda)\boldsymbol{a}\boldsymbol{a}^T
\end{equation}
\item \(A\) to be a probability matrix:
\begin{equation}
  A\boldsymbol{e}=\boldsymbol{d}
  \Rightarrow \mathrm{diag}(\boldsymbol{d})^{-1}A \rightarrow A
  % \Rightarrow A\leftarrow \mathrm{diag}(\boldsymbol{d})^{-1}A
\end{equation}
\end{itemize}
\end{frame}

\section{Numerical Examples}
\label{sec:org000d845}
\subsection{real-world data analysis}
\label{sec:orgc010065}
\begin{frame}[label={sec:org4fafe6b}]{Automobile Sales Data}
\begin{block}{Reference}
\citeauthor{Chiba_etal2017}
\citetitle{Chiba_etal2017}
\end{block}
\begin{itemize}
\item quarterly unit automobile sales data of manufacturers from
2007-1Q to 2015-4Q
\item estimate transition paths and discuss the relation between
social events and estimated results
\item objective:
\begin{equation}
  L(\{G_t\}) = \sum_t \|G_{t+1}-G_t\|_{1}
\end{equation}
\item optimization:
simplex method with slack variables
\end{itemize}
\end{frame}

\begin{frame}[label={sec:org422013b}]{}
\begin{center}
automobile sales for different manufactures\\
\includegraphics[page=4,
trim=40 285 30 80, clip,
width=.7\linewidth]
{Chiba_etal2017}
\end{center}
\end{frame}

\begin{frame}[label={sec:org1a32629}]{}
\begin{center}
market share transition\\
\includegraphics[page=7,
trim=40 90 30 425, clip,
width=.9\linewidth]
{Chiba_etal2017}
\end{center}
\end{frame}

\begin{frame}[label={sec:org16cb89a}]{}
\begin{center}
averages and standard deviations of sales shares\\
\includegraphics[page=8,
trim=40 365 30 70, clip,
width=.9\linewidth]
{Chiba_etal2017}
\end{center}
\end{frame}

\begin{frame}[label={sec:orgd728270}]{Graph Visualization}
\begin{columns}
\begin{column}{0.6\columnwidth}
\begin{center}
2007-1Q\\
\includegraphics[page=9,
trim=200 520 125 70, clip,
width=.9\linewidth]
{Chiba_etal2017}
\end{center}
\end{column}
\begin{column}{0.4\columnwidth}
\begin{itemize}
\item remove minor edge below 0.24
\item show market share with node size
\item cf. GM and Honda are allied
\end{itemize}
\end{column}
\end{columns}
\end{frame}

\begin{frame}[label={sec:org10bccef}]{}
\begin{center}
% 2007\\
\includegraphics[page=10,
trim=40 280 30 70, clip,
width=.7\linewidth]
{Chiba_etal2017}

{\footnotesize
  In March 2008, TOYOTA has become the world’s top seller by beating GM} 
\end{center}
\end{frame}

\begin{frame}[label={sec:org6313638}]{}
\begin{center}
% 2008/2009\\
\includegraphics[page=11,
trim=40 500 30 70, clip,
width=.9\linewidth]
{Chiba_etal2017}

{\footnotesize
  In 2009, TOYOTA launched a massive recall}
\end{center}
\end{frame}

\begin{frame}[label={sec:orgbf48da2}]{}
\begin{center}
% 2013\\
\includegraphics[page=11,
trim=40 110 30 460, clip,
width=.9\linewidth]
{Chiba_etal2017}

{\footnotesize
  In 2013, VW beats GM in total sales amount to claim
  second position in the automobile industry} 
\end{center}
\end{frame}

\section{Conclusion}
\label{sec:org46275fc}
\begin{frame}[label={sec:org953c6bd}]{Concluding Remarks}
we presented the followings

\begin{itemize}
\item a model of transitions and stationary distributions
\item a simple method for estimating transition matrices
from a sequence of stationary distributions
\item analysis of consumer transitions for sales share data
without detailed recording of consumer transitions
\end{itemize}

further investigation would be devoted to
\begin{itemize}
\item other objectives and constraints to improve the accuracy of
estimation and interpretability
\item other probabilistic models for estimating changes in
transitions
\end{itemize}
\end{frame}

\begin{frame}[allowframebreaks]{References}
\printbibliography[heading=none]
\end{frame}
\end{document}